%%%%%%%%%%%%%%%%%%%%%%%%%%%%%%%%%%%%%%%%%%%%
\documentclass[english,12pt]{article}
%\documentclass[aps,amsmath,amssymb,groupedaddress,twocolumn]{revtex4}
\usepackage{array}
\usepackage{graphicx}
\usepackage{amssymb}
\usepackage{amsmath}
\usepackage{multirow}
\usepackage{prettyref}
\usepackage{babel}
\usepackage{units}
\usepackage[latin1]{inputenc}
\usepackage{amsfonts}
\usepackage{amssymb}
\usepackage{babel}
\usepackage{color}
%\input{epsf}
\usepackage{cite}
\def\@fmsl@sh#1#2#3{\m@th\ooalign{$\hfil#1\mkern#2/\hfil$\crcr$#1#3$}}
\def\Ax{\mathcal{A}} \def\eq#1\en{\begin{equation}#1\end{equation}}
\def\s[#1,#2]{[#1\stackrel{\star}{,}#2]}
\def\sx[#1,#2]{[#1\stackrel{\star_{x}}{,}#2]} \def\pp#1{\partial_#1}
\def\lrhatDmu{\stackrel{\leftrightarrow}{\widehat D_\mu}}
\def\lrDmu{\stackrel{\leftrightarrow}{D_\mu}}
\def\lrDnu{\stackrel{\leftrightarrow}{D^\nu}}
\renewcommand{\baselinestretch}{1.2}

\textwidth 16.5cm
\textheight 655pt
\parskip 0.25cm
\hoffset -1.3cm 
\voffset -1.5cm
%\flushbottom

\newcommand{\nc}{\newcommand}
\nc{\beq}{\begin{equation}}
\nc{\eeq}{\end{equation}}
\nc{\beqa}{\begin{eqnarray}}
\nc{\eeqa}{\end{eqnarray}}
\def\DS {D\!\!\!\!/}
\def\A { A_\mu (x) }
\def\DM {\DS_{\, (\mu)}}
\def\O {{\cal O}}
\def\cx {\cal x}
\def\a {\alpha}
\def\bc{\begin{center}}
\def\ec{\end{center}}
\def\ie{{\it i.e.}}
\def\eg{{\it e.g.}}
\def\etc{{\it etc}}
\def\etal{{\it et al.}}
\def\ibid{{\it ibid}.}
\def\to{\rightarrow}
\def\lsim{\mathrel{\mathpalette\atversim<}}
\def\gsim{\mathrel{\mathpalette\atversim>}}
\def\DS {D\!\!\!\!/}
\def\A { A_\mu (x) }
\def\DM {\DS_{\, (\mu)}}
\def\O {{\cal O}}
\def\cx {\cal x}
\def\a {\alpha}
\def\bc{\begin{center}}
\def\ec{\end{center}}
\def\ts{\thinspace}
%\def\fpi{F_{\pi}}
\def\fpi{F}



\def\gsim{\mathrel{\rlap{\lower4pt\hbox{\hskip1pt$\sim$}}

    \raise1pt\hbox{$>$}}}       %greater than or approx. symbol



\def\ts{\thinspace}
%\def\fpi{F_{\pi}}
\def\fpi{F}

\def\gsim{\mathrel{\rlap{\lower4pt\hbox{\hskip1pt$\sim$}}
    \raise1pt\hbox{$>$}}}       %greater than or approx. symbol



%%%%%%%%%%%%%%%%%%%%

%%%%%%%%%%%%%%%%%%%%

\begin{document}
\makeatletter
\def\fmslash{\@ifnextchar[{\fmsl@sh}{\fmsl@sh[0mu]}}
\def\fmsl@sh[#1]#2{%
  \mathchoice
    {\@fmsl@sh\displaystyle{#1}{#2}}%
    {\@fmsl@sh\textstyle{#1}{#2}}%
    {\@fmsl@sh\scriptstyle{#1}{#2}}%
    {\@fmsl@sh\scriptscriptstyle{#1}{#2}}}
\def\@fmsl@sh#1#2#3{\m@th\ooalign{$\hfil#1\mkern#2/\hfil$\crcr$#1#3$}}
\makeatother
%\baselineskip 24pt

%%%%%%%%%%%%%%%%%%%%%%%%%%%%%%%%%%%%%%%%%%%%%%%%%%%%%%%%%%%%%%%%%
%%%
%%%                      TITLE PAGE
%%%
%%%%%%%%%%%%%%%%%%%%%%%%%%%%%%%%%%%%%%%%%%%%%%%%%%%%%%%%%%%%%%%%%
\thispagestyle{empty}
\begin{titlepage}
\boldmath
\begin{center}
  \Large {\bf  Symmetry and Option princing}
    \end{center}
\unboldmath
\vspace{0.2cm}
\begin{center}
{  {\large Xavier Calmet}\footnote{x.calmet@sussex.ac.uk}$^{a,b}$ and {\large Nathaniel Wiesendanger Shaw}\footnote{nmw24@sussex.ac.uk}$^{a}$}
 \end{center}
\begin{center}
$^a${\sl Department of Physics and Astronomy, 
University of Sussex, Brighton, BN1 9QH, United Kingdom
}\\
$^b${\sl PRISMA Cluster of Excellence and Mainz Institute for Theoretical Physics Johannes Gutenberg University, 55099 Mainz, Germany }
\end{center}
\vspace{5cm}
\begin{abstract}
\noindent
blabla
\end{abstract}  
\end{titlepage}

%\pacs{}


%%%%%%%%%%%%%%%%%%%%%%%%%%%%%%%%%%%%%%%%%%%%%%%%%%%%%%%%%%%%%%%%
%%%
%%%                     INTRODUCTION
%%%
%%%%%%%%%%%%%%%%%%%%%%%%%%%%%%%%%%%%%%%%%%%%%%%%%%%%%%%%%%%%%%%%

\newpage

Let me start from
\begin{eqnarray}
\label{original MG} \frac{\partial C}{\partial t}+ r S \frac{\partial C}{\partial S} + \frac{1}{2} V S^2 \frac{\partial^2 C}{\partial S^2} +(\lambda + \mu V) \frac{\partial C}{\partial V} +\rho \xi V^{1/2+\alpha} S \frac{\partial^2 C}{\partial S \partial V} +\xi^2 V^{2 \alpha}  \frac{\partial^2 C}{\partial V^2}= r C
\end{eqnarray}
$C=C(t,S,V)$, $V$ is time dependent.
 
The aim is to treat $S$ and $V$ as symmetrically as possible to make a global Galilean invariance in 2+1 manifest. Terms that violate this symmetry will be reintroduced as symmetry breaking terms. We will need to invent the rule to introduce these terms and not others if this is not the most generic symmetry breaking pattern. This is to be discussed later.

With this aim in mind, let me introduce an averaged volatility $\sigma^{2}$ (we can pick a standard way to define it) which is constant.  It is also clear that we need to pick $\alpha=1$  to emphasize the symmetry between $S$ and $V$. (By the way I am surprised that $\alpha$ can be changed like this, it will affect the dimension of this term, and something else must match its dimension).

Let me thus start from 
\begin{eqnarray}
\label{full merton} \frac{\partial C}{\partial t}+ r S \frac{\partial C}{\partial S} + \frac{1}{2} \sigma^{2} S^2 \frac{\partial^2 C}{\partial S^2} +\mu V \frac{\partial C}{\partial V} +\rho \xi V^{3/2} S \frac{\partial^2 C}{\partial S \partial V} +\xi^2 V^{2}  \frac{\partial^2 C}{\partial V^2}= r C
\end{eqnarray}
we will need to reintroduce $\frac{1}{2} V S^2 \frac{\partial^2 C}{\partial S^2}$,  $\lambda \frac{\partial C}{\partial V} $ and the terms corresponding to deviations from 1 for $\alpha$.

In the previous equation, we see that we have a mixed derivative term which we may not be able to diagonalize (we should check this again), I would thus set $\rho=0$ and reintroduce the term as symmetry breaking term. 

We thus end-up with:

\begin{equation}
\frac{\partial C}{\partial t} + rS\frac{\partial C}{\partial S} + \frac{1}{2}{\sigma^{2}}S^{2}\frac{\partial^2 C}{\partial S^2} + \mu V \frac{\partial C}{\partial V} + \xi^2 V^2 \frac{\partial^2 C}{\partial V^2} - rC = 0~,
\end{equation}


We now make the standard substitutions for the underlying and variance, treating them as equal spatial type variables: $x = \text{ln}(S)$ and $y = \text{ln}(V)$, yielding:

\begin{equation}
\frac{\partial C}{\partial t} + \kappa \frac{\partial C}{\partial x} + \theta \frac{\partial C}{\partial y} + \frac{1}{2}{\sigma^{2}}\frac{\partial^2 C}{\partial x^2} + \xi^2 \frac{\partial^2 C}{\partial y^2} - rC = 0~.
\end{equation}

Where, $\kappa = r - \frac{1}{2}{\sigma^{2}}$ and $\theta = \mu - \xi^2$.
\\
\indent In order to remove the constant term, $\frac{1}{2}{\sigma^{2}}$ outside the second derivative of x we make the time transformation: $\tau = \frac{{\sigma^{2}}}{2} (T-t)$, yielding: 

\begin{equation}
\label{pre-exponential} -\frac{\partial C}{\partial \tau} + (k - 1)\frac{\partial C}{\partial x} + \frac{2\theta}{{\sigma^{2}}}\frac{\partial C}{\partial y} + \frac{\partial^2 C}{\partial x^2} + \Theta\frac{\partial^2 C}{\partial y^2} - kC = 0~,
\end{equation}

Whereby: $k = \frac{2r}{{\sigma^{2}}}$ and $\Theta = \frac{2\xi^2}{{\sigma^{2}}}$. We proceed with the further substitution: $y = \frac{1}{\Theta}y$, transforming the coefficient of the second volatility related derivative to unity:

\begin{equation}
-\frac{\partial C}{\partial \tau} + (k - 1)\frac{\partial C}{\partial x} + \frac{\theta}{\xi^2}\frac{\partial C}{\partial y} + \frac{\partial^2 C}{\partial x^2} + \frac{\partial^2 C}{\partial y^2} - kC = 0~.
\end{equation}

\begin{equation}
C(x,y,\tau) = \phi(x,y,\tau)\psi(x,y,\tau)~,
\end{equation}

With: 

\begin{equation}
\phi(x,y,\tau) = \text{exp}(\alpha x + \beta y + \gamma\tau)~.
\end{equation}

Whereby the terms: $\alpha, \beta, \gamma$ are chosen by inspection, after substitution into \ref{pre-exponential} we see that the choice: 

%\begin{equation}
\begin{align}
\alpha &= -\frac{1}{2}(k - 1)~, \\
%\end{equation}
%\begin{equation}
\beta &= -\frac{\theta}{2\xi^2}~, \\
%\end{equation}
%\begin{equation}
\gamma &= -\frac{1}{4}(k + 1)^2 -\frac{\theta^2}{4\xi^4}~,
%\end{equation}
\end{align}

This yields \ref{pre-exponential} to become the heat equation in 2+1 dimensions:


\begin{equation}
\frac{\partial \psi}{\partial \tau} = \frac{\partial^2 \psi}{\partial x} + \frac{\partial^2 \psi}{\partial y^2}~.
\end{equation}





However, when we re-introduce the symmetry breaking terms, identified from \ref{full merton}, namely: $\frac{1}{2} V S^2 \frac{\partial^2 C}{\partial S^2}$,  $\lambda \frac{\partial C}{\partial V} $ and $\rho \xi V^{3/2} S \frac{\partial^2 C}{\partial S \partial V} $. We will re-introduce them as such:  $\frac{c_1}{2} V S^2 \frac{\partial^2 C}{\partial S^2}$, $c_2 \lambda \frac{\partial C}{\partial V}$, $c_{\rho} \xi V^{3/2} S \frac{\partial^2 C}{\partial S \partial V} $, where the constants $c_{i}$ control the magnitude of the symmetry breaking, yielding:

\begin{multline}
\frac{\partial \psi}{\partial \tau} = \frac{\partial^2 \psi}{\partial x^{2}}\Bigg{[}  1 +  \frac{c_{1}e^{y}}{2}  \Bigg{]} + \frac{\partial^2 \psi}{\partial y^2} + \frac{c_{\rho}\xi e^{y/2}}{\Theta}\frac{\partial^{2}\psi}{\partial x\partial y} - \frac{\partial \psi}{\partial x} \Bigg{[}\frac{ c_{\rho}\xi e^{y/2} \theta}{2\xi^{2}}  + k \frac{c_{1}e^{y}}{2}    \Bigg{]} \\ + \frac{\partial\psi}{\partial y} \Bigg{[}c_{2}\lambda\Theta e^{-y}   - \frac{c_{\rho}\xi e^{y/2}(k-1)}{2\Theta}   \Bigg{]}    + \psi \Bigg{[} \frac{c_{1}}{8}e^{y}(k^{2} - 1) - \frac{c_{2}\lambda\theta e^{-y}}{2\xi^{2}} +  c_{\rho}\xi e^{y/2}\frac{(k - 1)\theta}{4\xi^{2}}  \Bigg{]}~.
\end{multline}



\vskip15mm
\section{Dimensionality Check}
\vskip5mm

Here we check the dimensionality of the two terms in \ref{original MG} which contain $\alpha$. To do so we use the following notation: $[.]$ denotes the dimensions. First let us analyse the term: $\xi^2 V^{2 \alpha}  \frac{\partial^2 C}{\partial V^2}$

\begin{equation}
\Bigg{[} \xi^2 V^{2 \alpha}  \frac{\partial^2 C}{\partial V^2} \Bigg{]} = [\$] [\text{sec}^{2}][\text{sec}^{-4}][\text{sec}^{-1}]^{2\alpha}
\end{equation}

\begin{equation}
\Bigg{[} \xi^2 V^{2 \alpha}  \frac{\partial^2 C}{\partial V^2} \Bigg{]} = [\$] [\text{sec}^{-2}][\text{sec}^{-1}]^{2\alpha}
\end{equation}

\begin{equation}
\Bigg{[} \xi^2 V^{2 \alpha}  \frac{\partial^2 C}{\partial V^2} \Bigg{]} = [\$] [\text{sec}^{-1}]^{2(\alpha + 1)}
\end{equation}

From the above it is trivial to see that the dimensionality only equals $[\$][\text{sec}^{-1}]$, which is consistent with the other terms in \ref{original MG}, when $\alpha = -1/2$. Next we analyse: $\rho \xi V^{1/2+\alpha} S \frac{\partial^2 C}{\partial S \partial V}$

\begin{equation}
\Bigg{[} \rho \xi V^{1/2+\alpha} S \frac{\partial^2 C}{\partial S \partial V} \Bigg{]} = [\text{sec}][\text{sec}^{-2}][\text{sec}^{-1}]^{1/2+\alpha}[\$]
\end{equation}

\begin{equation}
\Bigg{[} \rho \xi V^{1/2+\alpha} S \frac{\partial^2 C}{\partial S \partial V} \Bigg{]} = [\text{sec}^{-1}]^{3/2+\alpha}[\$]
\end{equation}


The above is also consistent with the value of $\alpha = -1/2$. However, from solving: 

\begin{equation}
2(\alpha + 1) = \frac{3}{2} + \alpha~,
\end{equation}

We see that $\alpha = -1/2$ is the only solution for which the two terms have equal dimensions.

%%%%%%%%%%%%%%%%%%%%%%%%%%%%%%%%%%%%%%%%%%%%%%%%%%%%%%%%%%%%%%%%%
%%%
%%%                     COMMENTS
%%%
%%%%%%%%%%%%%%%%%%%%%%%%%%%%%%%%%%%%%%%%%%%%%%%%%%%%%%%%%%%%%%%%%



{\bf comments}

\begin{itemize}
\item we can parametrize the deviation from $\alpha=1$ as follows
$\xi^2 V^{2 \alpha}  \frac{\partial^2 C}{\partial V^2}\to\xi^2 V^{2}  \frac{\partial^2 C}{\partial V^2}- c_3 (\xi^2 V^{2}  \frac{\partial^2 C}{\partial V^2}-\xi^2 V^{2 \alpha}  \frac{\partial^2 C}{\partial V^2}) $ and $\rho \xi V^{1/2+\alpha} S \frac{\partial^2 C}{\partial S \partial V}\to \rho \xi V^{3/2} S \frac{\partial^2 C}{\partial S \partial V} 
- c_4(\rho \xi V^{3/2} S \frac{\partial^2 C}{\partial S \partial V} -
\rho \xi V^{1/2+\alpha} S \frac{\partial^2 C}{\partial S \partial V}) $ 
\item we need to look at the representations of the 2+1 Galilean group and check whether there are other useful quantities (vectors) that have a useful interpretation.
\item we need an interpretation of the symmetry breaking terms
\item we need to check dimensions of all parameters
\item We should solve (trivial) the symmetric 2+1 heat equation and transform back to standard variables the solution like one does for the BS model. This will give us some analytical solution which could be useful if it compares well to data/known solutions to full model with symmetry breaking term thus:
\item We should compare this solution to data / model with symmetry breaking terms that people have solved with other techniques
\item We should do perturbation theory around our analytical solution reintroducing the symmetry breaking terms and again compare to data.
\item Introduce the Lagrangian starting from the 2+1 heat equation and do the symmetry transformations back
\item introduce symmetry breaking terms in the lagrangian
\end{itemize}




Sanity check for my benefit:

looking at terms in
\begin{eqnarray}
\label{original MG} \frac{\partial C}{\partial t}+ r S \frac{\partial C}{\partial S} + \frac{1}{2} V S^2 \frac{\partial^2 C}{\partial S^2} +(\lambda + \mu V) \frac{\partial C}{\partial V} +\rho \xi V^{1/2+\alpha} S \frac{\partial^2 C}{\partial S \partial V} +\xi^2 V^{2 \alpha}  \frac{\partial^2 C}{\partial V^2}= r C
\end{eqnarray}
\begin{itemize}
\item $[C]=\$$
\item $[\frac{\partial C}{\partial t}]=\$/time$
\item $[r S \frac{\partial C}{\partial S}]= [r] \$ $ thus $[r]=1/time$
\item $[\frac{1}{2} V S^2 \frac{\partial^2 C}{\partial S^2}]=[V] \$$ thus  $[V]=1/time$
\item $[(\lambda + \mu V) \frac{\partial C}{\partial V}]=([\lambda] + [\mu] 1/time)$ thus $[\lambda]=\$/time$ and $[\mu]=\$$
\item $[\rho \xi V^{1/2+\alpha} S \frac{\partial^2 C}{\partial S \partial V}]= [\rho] [\xi] [1/time]^{1/2+\alpha} \$ time=\$ /time$ thus $[\rho] [\xi]= time^{\alpha-3/2}$
\item $[\xi^2 V^{2 \alpha}  \frac{\partial^2 C}{\partial V^2}]=[\xi]^2 [1/time]^{2 \alpha-2} \$=\$/time $ thus $[\xi]=time^{\alpha-3/2}$ and $\rho$ is dimensionless.
\end{itemize}



%%%%%%%%%%%%%%%%%%%%%%%%%%%%%%%%%%%%%%%%%%%%%%%%%%%%%%%%%%%%%%%%%
%%%
%%%                     ACKNOWLEDGEMENTS
%%%
%%%%%%%%%%%%%%%%%%%%%%%%%%%%%%%%%%%%%%%%%%%%%%%%%%%%%%%%%%%%%%%%%



{\it Acknowledgments:}
The work of XC is supported in part  by the Science and Technology Facilities Council (grant number  ST/P000819/1). XC is very grateful to MITP for their generous hospitality during the academic year 2017/2018. 

%%%%%%%%%%%%%%%%%%%%%%%%%%%%%%%%%%%%%%%%%%%%%%%%%%%%%%%%%%%%%%%%%
%%%
%%%                     BIBLIOGRAPHy
%%%
%%%%%%%%%%%%%%%%%%%%%%%%%%%%%%%%%%%%%%%%%%%%%%%%%%%%%%%%%%%%%%%%%


\bigskip{}


\baselineskip=1.6pt 

\begin{thebibliography}{10}

   
\end{thebibliography}



\end{document}

